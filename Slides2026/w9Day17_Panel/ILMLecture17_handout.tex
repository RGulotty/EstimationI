
\documentclass[aspectratio=169]{beamer}

%\usepackage[table]{xcolor}
\mode<presentation> {
\setbeamercovered{transparent}
  \usetheme{Boadilla}

%  \usetheme{Pittsburgh}
%\usefonttheme[2]{sans}
\renewcommand{\familydefault}{cmss}
%\usepackage{lmodern}
%\usepackage[T1]{fontenc}
%\usepackage{palatino}
%\usepackage{cmbright}
\usepackage{bm}
\usepackage{listings}
\useinnertheme{rectangles}
}
\usepackage{amsmath}
\setbeamercolor{normal text}{fg=black}
\setbeamercolor{structure}{fg= blue}
\definecolor{trial}{cmyk}{1,0,0, 0}
\definecolor{trial2}{cmyk}{0.00,0,1, 0}
\definecolor{darkgreen}{rgb}{0,.4, 0.1}
\definecolor{darkpurple}{rgb}{0.4, 0, 0.6}
\usepackage{array}
\usepackage{booktabs}
\beamertemplatesolidbackgroundcolor{white}  \setbeamercolor{alerted
text}{fg=darkpurple}
\usepackage{tcolorbox}
\setbeamertemplate{caption}[numbered]\newcounter{mylastframe}

\font\domino=domino
\def\die#1{{\domino#1}}
\usepackage{tikz}
\usetikzlibrary{arrows}
\usepackage{colortbl}

\renewcommand{\familydefault}{cmss}
%\usepackage[all]{xy}

\usepackage{tikz}
\usepackage{lipsum}

 \newenvironment{changemargin}[3]{%
 \begin{list}{}{%
 \setlength{\topsep}{0pt}%
 \setlength{\leftmargin}{#1}%
 \setlength{\rightmargin}{#2}%
 \setlength{\topmargin}{#3}%
 \setlength{\listparindent}{\parindent}%
 \setlength{\itemindent}{\parindent}%
 \setlength{\parsep}{\parskip}%
 }%
\item[]}{\end{list}}
\usetikzlibrary{arrows}

\lstset{%
  language=R,
  basicstyle=\ttfamily\small,
  keywordstyle=\color{blue},
  commentstyle=\color{darkgreen},
  stringstyle=\color{darkpurple},
  showstringspaces=false,
  breaklines=true,
  frame=single,
  backgroundcolor=\color{gray!10}
}

\usecolortheme{lily}

\newtheorem{com}{Comment}
\newtheorem{lem} {Lemma}
\newtheorem{prop}{Proposition}
\newtheorem{condition}{Condition}
\newtheorem{thm}{Theorem}
\newtheorem{defn}{Definition}
\newtheorem{cor}{Corollary}
\newtheorem{obs}{Observation}
 \numberwithin{equation}{section}
 
\makeatletter
\def\beamerorig@set@color{%
  \pdfliteral{\current@color}%
  \aftergroup\reset@color
}
\def\beamerorig@reset@color{\pdfliteral{\current@color}}
\makeatother
\setbeamertemplate{navigation symbols}{}

\useoutertheme{miniframes}
\title[PLSC 30700]{Linear Models Lecture 17: Pooled and Panel Data}

\author{Robert Gulotty}
\institute[Chicago]{University of Chicago}
\vspace{0.3in}


\begin{document}



\begin{frame}
\maketitle
\end{frame}
\section{Intro}
\setcounter{subsection}{1}


\begin{frame}{Where We Are: From GMM to Panel Data}
\begin{itemize}
\item In Lectures 15--16 we developed \textbf{GMM} as a unified framework:
\begin{itemize}
\item Specify moment conditions $E[\bm{g}(\bm{z}_i,\bm{\theta}_0)]=\bm{0}$
\item Estimate $\hat{\bm{\theta}}$ by minimizing $\hat{\bm{g}}'\bm{W}\hat{\bm{g}}$
\item Test overidentifying restrictions with the $J$-statistic
\end{itemize}\pause
\item \textbf{Panel data} introduces new structure:
\begin{itemize}
\item Repeated observations create natural moment conditions
\item Unobserved heterogeneity $\alpha_i$ must be controlled
\item Different assumptions $\Rightarrow$ different moment conditions $\Rightarrow$ different estimators
\end{itemize}\pause
\item \textbf{This lecture:} pooled OLS, fixed effects, between estimators as moment-based estimators.\\
\textbf{Next lecture:} random effects, CRE, and dynamic panel GMM (Arellano-Bond).
\end{itemize}
\end{frame}

\begin{frame}{Panel Estimators as Moment Conditions}
\centering
\small
\begin{tabular}{@{}llp{3.8cm}@{}}
\toprule
\textbf{Estimator} & \textbf{Moment Condition} & \textbf{Key Assumption} \\
\midrule
Pooled OLS & $E[\bm{X}_{it}' e_{it}]=\bm{0}$ & $\alpha_i$ uncorrelated with $\bm{X}$ \\[4pt]
Fixed Effects & $E[\tilde{\bm{X}}_{it}'\tilde{e}_{it}]=\bm{0}$ & Strict exogeneity (within-unit) \\[4pt]
Random Effects & $E[\bm{X}_{it}'\nu_{it}]=\bm{0}$ & $\alpha_i$ uncorrelated with $\bm{X}$; GLS weighting \\[4pt]
Arellano-Bond & $E[Y_{is}\cdot\Delta e_{it}]=0,\; s\leq t\!-\!2$ & Sequential exogeneity; lagged levels as instruments \\
\bottomrule
\end{tabular}
\vspace{6pt}

\begin{itemize}
\item Each row is a \textbf{GMM estimator} with different instruments and assumptions.\pause
\item More moment conditions $\Rightarrow$ overidentification $\Rightarrow$ testable with the $J$-test.
\end{itemize}
\end{frame}

\begin{frame}{Longitudinal Data}
\begin{itemize}
\item Longitudinal data describes two dimensions, usually time and individuals.
\item Examples: 
\begin{itemize}
\item Exit Polls/ CPS/ ANES: people are surveyed following each election.
\item Dyadic data in peace science literature.
\item Election forecasting.
\end{itemize}
\item Data can be stacked to make a data matrix with $N_1+N_2+\cdots +N_T$ rows.
\item This structure can be incorporated into our statistical model.
\end{itemize}
\end{frame}

\begin{frame}{Using time in our analysis}
\begin{itemize}
\item Time can be treated as an indicator variable (as dummy variables or cumulatively), or as a continuous variable.
\item Time period interactions can evaluate changes in slope coefficients across time (the effect of a free fedora on employment may be different today than in 1940).
\end{itemize}
\end{frame}


\section{Difference in Difference}
\setcounter{subsection}{1}

\begin{frame}{Differences in Differences}
\begin{itemize}
\item We might collect data over time to rule out certain confounding factors.\pause
\item Consider a study of the effect (D) of a voter ID law on turnout (Y), observing Texas and Oklahoma before and after the law.\pause
\begin{itemize}
\item Comparison one:\\
\begin{tabular}{lll}
State&&Outcome\\
\hline
Texas && $Y = M+D$\\
Oklahoma  && $Y = O$\\
\end{tabular}\pause
\item Comparison two:\\
\begin{tabular}{lll}
State&Time&Outcome\\
\hline
Texas& Before & $Y = M$\\
& After & $Y = M+(T+D)$\\
\end{tabular}
\end{itemize}
\item Neither comparison on its own recovers D.
\end{itemize}
\end{frame}


\begin{frame}{Diff in Diff}

\begin{tabular}{lllll}
State&Time& Outcome& After-Before& dTexas-dOklahoma\\
\hline
Texas & Before&$Y = M$\\
 & After&$Y = M+T+D$& T+D\\
 \\
Oklahoma  &Before& $Y = O$\\
& After & $Y = O+T$&T\\
&&&&D
\end{tabular}
$$D=dY_{T}-dY_{O}=(Y_{A,T}-Y_{B,T})-(Y_{A,O}-Y_{B,O})$$
\end{frame}

\begin{frame}{Potential Outcomes}
\begin{itemize}
\item Define $D$ as the binary treatment $d\in\{0,1\}$ which tells us if we will be in the treatment category once we get to the second period.
\item Measurements happen twice, $t\in\{0,1\}$
\item $Y_t^d$ is the outcome in period $t$ for treatment status $d$.
\item The outcomes we can observe is $Y_t$ (no d).
$$Y_t=dY_t^1+(1-d)Y^0_t$$
\item Counterfactual outcomes: 
\begin{itemize}
\item $Y_t^1|D_t=0$  (Outcomes in a world of treatment among those who weren't)
\item $Y_t^0|D_t=1$ (Outcomes in a world of control among those who were treated)
\end{itemize}
\end{itemize}
$$\delta_{i}=Y_{t}^1-Y_{t}^0$$
\end{frame}

\begin{frame}{Parallel Trends}
\begin{itemize}
\item Identification assumption/Parallel Trends
$$E[Y_{t=1}^0|D=1]-E[Y_{t=0}^0|D=1]=E[Y_{t=1}^0|D=0]-E[Y_{t=0}^0|D=0]$$
\item In previous example, T is the same across states, state level differences $(M-O)$ don't vary over time.
\item The differences in expected potential non-treatment outcomes are unrelated to belonging to the treated or control group in the post treatment period.
\item If the treated had not been subjected to treatment, both subpopulations would have experienced the same time trends.
\item The composition cannot be affected by treatment
\end{itemize}
\end{frame}



\begin{frame}{Example Obama Support}
\begin{itemize}
\item Suppose we want to study the effect of racial co-identification (non-identification) on feeling thermometers of Obama in 2008.
$$ObamaRating_i=b_0+d_{08} Black_i+\bm{b'_x}\bm{x}_i+e_i$$\pause
\item However, we are worried about the two kinds of omitted factors:
\begin{itemize}\pause
\item Time trends (T): Obama ran after a financial crisis and war that might make him popular generally.\pause
\item Group Differences (M-O): Black and white people might differ in their preferences over any president that is a relatively liberal northern antiwar Democratic senator. \pause
\end{itemize}
\item Suppose Obama's appeal (repulsion) among Black (white) people depends on the economic conditions in 2008, his being a liberal Senator, and not only his race?
\end{itemize}
\end{frame}

\begin{frame}{Example Obama Support}
\begin{itemize}
\item We can use a repeated cross section to difference out similar attitudes toward another Democratic candidate.\pause
$$Kerry_i=b_0+d_{04} Black_i+\bm{b'_x}\bm{x}_i+e_i$$
\item If we assume that 
\begin{itemize}
\item All time varying factors other than the candidate's race affected Black and white people's response to Democratic nominees equally.\pause
\item All differences in evaluations of D. nominees between Black and white people did not change between 04 and 08.\pause
\end{itemize}
\item Then $\delta$ is an estimate of the Average Treatment Effect of the candidate's race on attitudes.
$$\delta=d_{08}-d_{04}=(\bar{r}_{08,B}-\bar{r}_{08,W})-(\bar{r}_{04,B}-\bar{r}_{04,W})$$
\end{itemize}
\end{frame}

\begin{frame}{Estimating Differences in Differences}
\begin{itemize}
\item Stack 04 and 08 data and compare two linear models:
\item Define $yr08_i=1$ for the 2008 survey takers and $B_i=\mathbb{I}(\text{Race="Black"})$\pause
\begin{align*}
DRating_i&=b_{0,08}yr08_i+b_{0,04}(1-yr08_i)+d_{08}yr08_i B_i+d_{04}(1-yr08_i) B_i+\bm{b'_x}\bm{x}_i+e_i\\
DRating_i&=b_{0,08}yr08_i+b_{0,04}(1-yr08_i)+d B_i+\bm{b'_x}\bm{x}_i+e_i
\end{align*}\pause
\item Then an F test tells us the effect of racial co-identification or non-identification.
\end{itemize}
\end{frame}

\begin{frame}

\scriptsize{\begin{table}[!htbp] \centering 
\begin{tabular}{@{\extracolsep{5pt}}lccc} 
\\[-1.8ex]\hline 
\\[-1.8ex] & \multicolumn{3}{c}{demft} \\
\hline \\[-1.8ex] 
 I(1 - y08) & 54.5$^{***}$ & 53.0$^{***}$ &  \\ 
  & (1.4) & (1.4) &  \\ 
  & & & \\ 
y08 & 65.2$^{***}$ & 65.9$^{***}$ & 10.6$^{***}$ \\ 
  & (1.3) & (1.3) & (1.0) \\ 
  & & & \\ 
 black &  & 21.7$^{***}$ & 12.8$^{***}$ \\ 
  &  & (1.0) & (1.9) \\ 
  & & & \\ 
 I(1 - y08):black & 12.8$^{***}$ &  &  \\ 
  & (1.9) &  &  \\ 
  & & & \\ 
 y08:black & 25.1$^{***}$ &  & 12.3$^{***}$ \\ 
  & (1.2) &  & (2.3) \\ 
  & & & \\ 
   age & 0.02 & 0.02 & 0.02 \\ 
  & (0.02) & (0.02) & (0.02) \\ 
  & & & \\ 
 Constant &  &  & 54.5$^{***}$ \\ 
  &  &  & (1.4) \\ 
\hline 
\hline \\[-1.8ex] 
\end{tabular} 
\end{table} }
\end{frame}


\begin{frame}
\begin{center}
\includegraphics[width= 5 in]{ObamaDiD.png}
\end{center}
\end{frame}

\begin{frame}{Policy Evaluation}
\begin{itemize}
\item Diff-in-diff is often used to study binary policies when we theorize that selection bias is driven by fixed group features.\pause
\item Key assumption for causal inference: parallel trends:
\begin{itemize}
\item The amount of uncontrolled selection bias is not changing over time.
\item Time trends are the same across groups.
\end{itemize}\pause
\item Example of violation of parallel trends: Suppose that Texas and Oklahoma are trending blue, but Texas is closer to flipping, causing both voter ID laws and more investment by Democrats.
\end{itemize}
\end{frame}


\begin{frame}
\begin{center}
\includegraphics[width= 4.5 in]{DIDgraph.png}
\end{center}
\end{frame}


\begin{frame}{Alternative: Synthetic Control}
\begin{itemize}
\item When we are studying aggregates like states, we might not have parallel trends.
\item However, in some cases we can construct a plausible counterfactual from combinations of untreated units.
\item The synthetic control approach uses covariates to construct these counterfactual weighted combinations.
\item Assumes that we have a long time series and the treated unit outcomes lie within the convex hull of untreated outcomes.
\end{itemize}
\end{frame}


\section{Panel Data}
\setcounter{subsection}{1}


\begin{frame}{Panel data}
\begin{itemize}
\item The above analysis assumes that group compositions don't change.
\item Suppose we can measure the same individuals.
\item We can then account for variation within groups, across groups, and across time.
\end{itemize}
\end{frame}


\begin{frame}{Panel Example 1: Voting behavior}
\begin{itemize}
\item Consider a population of families.  Choose one at random.\pause
\item For each family member $t$, we observe voting decisions $Y_t$ and exposure to a persuasive messages $Z_t$.\pause
\item Each family has a household wealth $W$ and a set of stories passed down from their grandparents $A$.\pause
\item We observe $Y_t$, $Z_t$ and $W$, but not $A$. \pause
\item We want to estimate the effect of $Z_t$ on $Y_t$, holding constant $W$ and $A$.
\end{itemize}
\end{frame}

\begin{frame}{Panel Example 2: Airline cost structure}
\begin{itemize}
\item Consider a population of airlines. Choose one at random.\pause
\item In each year, $t$, each airline pays $Y_t$ dollars to fly $Z_t$ passengers.\pause
\item Each airline uses a proprietary technology $A$ which does not vary over time.\pause
\item We observe N randomly selected Airlines but do not measure $A$. \pause
\item We want to estimate the way that $Y_t$ responds to $Z_t$, holding constant $A$.
\end{itemize}
\end{frame}

\begin{frame}{Panel Example 3: Candidate Electoral Performance}
\begin{itemize}
\item Consider a population of candidates. Choose one at random.
\item For each year $t$, the candidate receives vote share $Y_t$.
\item We know each candidate has an underlying charisma $A$ which does not vary over time.
\item We observe N randomly selected candidates, but do not measure $A$. 
\item We want to estimate the effect of $Y_{t-1}$ on $Y_t$, holding constant $A$.
\end{itemize}
\end{frame}


\begin{frame}{Panel Data}
\begin{itemize}
\item The voters in a family, the airlines and candidates over time are all examples of \emph{panel data}.\pause
\item Our observations are double indexed, usually divided into $N$ units and $T$ time periods/groups. \pause
\item If there are many $T$ but few $N$, we call the data long and narrow, (time series cross section).\pause
\item If every unit is observed in every period/ group, the data are called balanced.
\end{itemize}
\end{frame}

\begin{frame}{Example Long Narrow Data }
\begin{itemize}
\item Greene analyzes the cost structure in the airline industry from 1970-84.
\item Research question: do airlines have positive economies of scale?
\item library(AER); data("USAirlines"); analyzes determinants of total costs for 6 airlines.\pause
\item The cost function follows a Cobb-Douglas specification. Taking logs yields a linear regression:
$$\ln(\text{cost})= \beta_0+\beta_1\ln(\text{output})+\beta_2\ln(\text{fuel price})+\beta_3\text{load}+u$$
\item \emph{cost}: total costs of labor materials etc in USD 1000.
\item \emph{output}: revenue passenger mile index
\item input \emph{prices}: fuel prices
\item \emph{load}: share of seats sold on flights flown.
\end{itemize}
\end{frame}


\section{Pooled Data}
\setcounter{subsection}{1}



\begin{frame}{Example Long Narrow Data: Airline cost}
\begin{center}
\includegraphics[width= 5 in]{airpanel.png}
\end{center}
\end{frame}



\begin{frame}{Pooled Regression of log(costs)}
\begin{table}
\begin{center}
\begin{tabular}{l c c}
\hline
 & Greene 2000& Greene 2003 \\
\hline
(Intercept)  & $9.52^{***}$  & $9.42^{***}$  \\
             & $(0.23)$      & $(0.23)$      \\
lnoutput     & $0.88^{***}$  & $0.94^{***}$  \\
             & $(0.01)$      & $(0.03)$      \\
lnpfuel      & $0.45^{***}$  & $0.46^{***}$  \\
             & $(0.02)$      & $(0.02)$      \\
load         & $-1.63^{***}$ & $-1.54^{***}$ \\
             & $(0.35)$      & $(0.34)$      \\
lnoutput$^2$ &               & $0.02^{*}$    \\
             &               & $(0.01)$      \\
\hline
R$^2$        & $0.99$        & $0.99$        \\
Num. obs.    & $90$          & $90$          \\
\hline
\end{tabular}
\end{center}
\end{table}
\end{frame}

\begin{frame}{Cost and Output (Bivariate)}
\includegraphics[width= 5 in]{airgrandmean0.png}
\end{frame}

\begin{frame}{Cost and Output (Bivariate)}
\includegraphics[width= 5 in]{airgrandmean.png}
\end{frame}


\begin{frame}{Cost and price of fuel (Bivariate)}
\includegraphics[width= 5 in]{airgrandmean2.png}
\end{frame}

\begin{frame}{Cost and Capacity Utilization (load)}
\includegraphics[width= 5 in]{airgrandmean3.png}
\end{frame}

\begin{frame}{resid(Cost) and resid(load)}
\includegraphics[width= 5 in]{airgrandmean4.png}
\end{frame}


\begin{frame}{Assumptions of Pooled Regression: Exogeneity}
\begin{itemize}
\item $\bm{y}=\bm{X}\bm{\beta}+\bm{e}$
\item Strict Exogeneity: $E(e_{it}|\bm{X})=\bm{0}$ for all time periods of $\bm{X}$.
\item That is, $\bm{X}$ must not have any long run consequences for $\bm{Y}$.
\item Moreover, $\bm{X}$ must not itself be a function of past $\bm{Y}$ 
\end{itemize}
\end{frame}

\begin{frame}{Autocorrelation versus Autoregression}
\begin{align*}
Y_t&=\alpha Y_{t-1}+\beta X_t+\nu_t\\
\nu_t&=\rho \nu_{t-1}+e_t
\end{align*}
\begin{itemize}\pause
\item $\rho\neq 0$, $\alpha = 0$ has a "common factor" problem (autocorrelation): 
\begin{itemize}
\item Outcomes at $t-1$ are associated with outcomes at $t$ for many unobserved reasons.
\item OLS slope estimates are unbiased but inefficient and produces wrong standard errors.
\item Addressed with GLS/ serial correlation robust standard errors. \pause
\end{itemize}
\item $\alpha \neq 0$ is associated with dynamics: 
\begin{itemize}
\item Outcomes at $t-1$ affect outcomes now.
\item OLS is biased, inefficient and produces wrong standard errors. 
\item Addressed with lagged variables and other structural time series approaches.
\end{itemize}
\end{itemize}
\end{frame}


\begin{frame}{Improvements on Inference}
\begin{itemize}
\item In panel data we likely violate of iid errors, as observations from the same group/firm are likely associated
$$E[e_{ig}e_{jg}]=\rho\sigma^2_e>0 $$
\item Here $\rho$ is the intra-group/firm correlation.
\end{itemize}
\end{frame}
%
%\begin{frame}{How far off are we?}
%\begin{itemize}
%\item Consider an additive model of the residuals.
%$$e_{ig}=\nu_g+\eta_{ig}$$\pause
%\item Where $v_g$ is a random shock for group g, and $\eta_{ig}$ is the remaining individual level component.\pause
%\item Given this error structure: $\rho=\frac{\sigma^2_\nu}{\sigma^2_\nu+\sigma^2_\eta}$\pause
%\item With the same sample sizes across time,
%$$\frac{V(\hat{\beta}_1)}{V_{regular}(\hat{\beta}_1)}=1+(n-1)\rho$$\pause
%\item If $\rho=1$, then $V_{regular}(\hat{\beta}_1)$ should be $n$ times larger!
%\end{itemize}
%\end{frame}


\begin{frame}{Standard error problems.}
\begin{itemize}
\item If the number of groups is large, we can cluster standard errors using a formula similar to White standard errors.\pause
\item If the number of groups is small, we can just inflate our standard errors by $\sqrt{1+(n-1)\rho}$ (the Moulton factor)\pause
\item There is a bootstrap solution that has good performance
\end{itemize}
\end{frame}


\begin{frame}{White vs Cluster Robust}
\begin{table}
\begin{center}
\begin{tabular}{l c c}
\hline
 & (white) & (cluster by airline) \\
\hline
(Intercept) & $9.52^{***}$  & $9.52^{***}$ \\
            & $(0.22)$      & $(0.38)$     \\
lnoutput    & $0.88^{***}$  & $0.88^{***}$ \\
            & $(0.01)$      & $(0.02)$     \\
lnpfuel     & $0.45^{***}$  & $0.45^{***}$ \\
            & $(0.02)$      & $(0.03)$     \\
load        & $-1.63^{***}$ & $-1.63^{*}$  \\
            & $(0.32)$      & $(0.44)$     \\
\hline
R$^2$       & $0.99$        & $0.99$       \\
N Clusters  & $$            & $6$          \\
\hline
\end{tabular}
\end{center}
\end{table}
\end{frame}

\section{Between and Within Estimators}
\setcounter{subsection}{1}

\begin{frame}{Within-group estimators}
\begin{itemize}
\item Consider the typical panel model: $y_{it}=\gamma x_{it}+f_i+e_{it}$\pause
\item Assume that $E[e_i|f_i, \bm{x}_{i1},\bm{x}_{i2},\ldots \bm{x}_{iT}]=0\ \quad \forall i$ \pause
\item But suppose that $E(f_i|x_{it})\neq 0$.  We can dummy for "fixed effects" $f$, producing
$$\bm{y}=\bm{X\beta}+\bm{Df}+\bm{e}$$\pause
\item We know from Frisch Waugh Lovell that:
\begin{align*}
\hat{\bm{\beta}}&=(\bm{X}'\bm{M}_D\bm{X})^{-1}\bm{X}'\bm{M}_D\bm{y}\\
\hat{\bm{f}}&=(\bm{X}'\bm{M}_X\bm{X})^{-1}\bm{X}'\bm{M}_X\bm{y}
\end{align*}\pause
\item Where $\bm{M}_X=\bm{I}-\bm{X}'(\bm{X}'\bm{X})^{-1}\bm{X}'$ and $\bm{M}_D=I-\bm{D}'(\bm{D}'\bm{D})^{-1}\bm{D}'$  are the idempotent residual makers.
\item $\hat{\beta}$ and $\hat{f}$ are the "within-group" estimators.
\end{itemize}
\end{frame}

\begin{frame}
\begin{itemize}
\item Suppose N=2: $\bm{D}=\begin{pmatrix} \bm{i}&\bm{0}\\\bm{0}&\bm{i}\end{pmatrix}$ where $\bm{i}$ is a ($T\times 1$) vector of 1s\pause
\begin{align*}
\bm{M}_D&=\bm{I}-\bm{D}(\bm{D}'\bm{D})^{-1}\bm{D}'\\\pause
&=\begin{pmatrix} \bm{I}&\bm{0}\\\bm{0}&\bm{I}\end{pmatrix}-\begin{pmatrix} \bm{i}&\bm{0}\\\bm{0}&\bm{i}\end{pmatrix}\left[\begin{pmatrix} \bm{i}'&\bm{0}\\\bm{0}&\bm{i}'\end{pmatrix}'\begin{pmatrix} \bm{i}&\bm{0}\\\bm{0}&\bm{i}\end{pmatrix}\right]^{-1}\begin{pmatrix} \bm{i}'&\bm{0}\\\bm{0}&\bm{i}'\end{pmatrix}\\\pause
&=\begin{pmatrix} \bm{I}&\bm{0}\\\bm{0}&\bm{I}\end{pmatrix}-\begin{pmatrix} \bm{i}&\bm{0}\\\bm{0}&\bm{i}\end{pmatrix}\begin{pmatrix}T&\bm{0}\\\bm{0}&T\end{pmatrix}^{-1}\begin{pmatrix} \bm{i}'&\bm{0}\\\bm{0}&\bm{i}'\end{pmatrix}\\\pause
&=\begin{pmatrix} \bm{I}&\bm{0}\\\bm{0}&\bm{I}\end{pmatrix}-\begin{pmatrix} \bm{i}&\bm{0}\\\bm{0}&\bm{i}\end{pmatrix}\begin{pmatrix} 1/T&\bm{0}\\\bm{0}&1/T\end{pmatrix}\begin{pmatrix} \bm{i}'&\bm{0}\\\bm{0}&\bm{i}'\end{pmatrix}\\\pause
&=\begin{pmatrix} \bm{I}-\frac{1}{T}\bm{i}\bm{i}'&\bm{0}\\\bm{0}&\bm{I}-\frac{1}{T}\bm{i}\bm{i}'\end{pmatrix}
\end{align*}\pause 
\end{itemize}
\end{frame}



\begin{frame}
\begin{align*}
\bm{M}_D\bm{y}&=\begin{pmatrix} \bm{I}-\frac{1}{T}\bm{i}\bm{i}'&\bm{0}\\\bm{0}&\bm{I}-\frac{1}{T}\bm{i}\bm{i}'\end{pmatrix}\bm{y}\\\pause
&=\begin{pmatrix} \bm{I}-\frac{1}{T}\bm{i}\bm{i}'&\bm{0}\\\bm{0}&\bm{I}-\frac{1}{T}\bm{i}\bm{i}'\end{pmatrix}\begin{pmatrix}\bm{y}_1\\ \bm{y}_2 \end{pmatrix}\\
&=\begin{pmatrix} y_{11}-\frac{\sum_{t=1}^T Y{_1t}}{T}\\
 y_{12}-\frac{\sum_{t=1}^T Y{_1t}}{T}\\
 \ldots\\
  y_{1T}-\frac{\sum_{t=1}^T Y{_1t}}{T}\\
    y_{21}-\frac{\sum_{t=1}^T Y{_2t}}{T}\\
        y_{22}-\frac{\sum_{t=1}^T Y{_2t}}{T}\\
         \ldots\\
            y_{2T}-\frac{\sum_{t=1}^T Y{_2t}}{T}
            \end{pmatrix}
\end{align*}
\end{frame}

\begin{frame}{Within Estimator}
Two equivalent ways to estimate:
\begin{enumerate}
\item Explicitly controlling for fixed effects $\bm{f}$ and estimating $\hat{\bm{\beta}}=(\bm{X}'\bm{M}_D\bm{X})^{-1}\bm{X}'\bm{M}_D\bm{y}$.\pause
\item Differencing out the group means:\\
\end{enumerate}
$$y_{it}-\bar{y}_i=\bm{\beta}'(\bm{x}_{it}-\bar{\bm{x}}_i)+e_{it}-\bar{e}_i$$
$$\bm{b}_W=(\bm{S}_{xx}^W)^{-1}\bm{S}_{xy}^W$$
$$\bm{S}_{xx}^W= \sum_{i=1}^N\sum_{t=1}^T(\bm{x}_{it}-\bar{\bm{x}}_i)(\bm{x}_{it}-\bar{\bm{x}}_i)'$$
$$\bm{S}_{xy}^W= \sum_{i=1}^N\sum_{t=1}^T(\bm{x}_{it}-\bar{\bm{x}}_i)(y_{it}-\bar{y}_i)'$$
\end{frame}




\begin{frame}{Aside on Stacked Matrix Notation}
$\bm{y}_i$: $T\times 1$, $\bm{X}_i$:  $T\times k$, $\bm{e}_i$: $T\times 1$, $\bm{\beta}:$ $k\times 1$
\pause
\begin{align*}
\begin{bmatrix}\bm{y}_1\\\bm{y}_2\\ \vdots \\ \bm{y}_N \end{bmatrix}&=\begin{bmatrix}\bm{X}_1\\\bm{X}_2\\ \vdots \\ \bm{X}_N \end{bmatrix} \bm{\beta}+\begin{bmatrix}\alpha_1 \bm{1}\\\alpha_2 \bm{1}\\ \vdots \\ \alpha_N \bm{1} \end{bmatrix}+\begin{bmatrix}\bm{e}_1\\\bm{e}_2\\ \vdots \\ \bm{e}_N \end{bmatrix}\\\pause
y&= \bm{X}\bm{\beta}+\bm{\alpha}\otimes \bm{1}_T+\bm{e}
\end{align*}
\end{frame}


\begin{frame}{Software Implementations}
\begin{itemize}
\item lm(y$\sim x+f_1+f_2+\ldots +f_n)$
\item plm(y$\sim x$, c("unit", "time")
\item fixest::feols(y$\sim$ x$|$ fe)
\item You can also demean by time averages, but you need to correct the standard errors.
\end{itemize}
\end{frame}


\begin{frame}{Fixed Effects and unobserved factors}
Just as with difference in difference, fixed effects address unobserved effects ($\alpha_i$).
$$y_{it}=\gamma x_{it}+f_i+\alpha_i+e_{it}$$
The fixed effects transformation again mean deviates the variables, giving us:
$$y_{it}-\bar{y}_i=\bm{\beta}'(\bm{x}_{it}-\bar{\bm{x}}_i)+\alpha_i-\bar{\alpha}_i+e_{it}-\bar{e}_i$$
\end{frame}



\begin{frame}{Cost and Output}
\includegraphics[width= 5 in]{airgrandmean.png}
\end{frame}

\begin{frame}{Cost and Output (within estimator)}
\includegraphics[width= 5 in]{airgrandmeanFe.png}
\end{frame}

\begin{frame}{Small note on estimation}
$$\bm{b}_{fe}=\bm{\beta}+(\bm{X}'\bm{M}_D\bm{X})^{-1}\bm{X}'\bm{M}_D\bm{e}$$
Under iid errors, we have:
$var(\bm{b}_{fe})=\sigma_{e}^2(\bm{X}'\bm{M}_D\bm{X})^{-1}$.
But we need to correct our estimator for the $N$ estimated means:
$$s_e^2=\frac{\sum_{i=1}^N\sum_{t=1}^T e^2_{it}}{NT-N-K}$$
\end{frame}

\begin{frame}{Fixed Effects as GMM Moment Conditions}
\begin{itemize}
\item Define the within-transformed variables $\tilde{Y}_{it}=Y_{it}-\bar{Y}_i$ and $\tilde{\bm{X}}_{it}=\bm{X}_{it}-\bar{\bm{X}}_i$.
\item The FE (within) estimator solves the moment conditions:
$$E[\tilde{\bm{X}}_{it}'\tilde{e}_{it}]=\bm{0}$$\pause
\item This is a \textbf{just-identified} GMM problem: $k$ moment conditions for $k$ parameters.
\begin{itemize}
\item No overidentifying restrictions $\Rightarrow$ no $J$-test with FE alone.
\end{itemize}\pause
\item \textbf{Preview:}
\begin{itemize}
\item Random effects adds between-group moments $\Rightarrow$ more efficient under stronger assumptions.
\item Arellano-Bond (next lecture) creates \emph{many} moment conditions from lagged levels $\Rightarrow$ overidentification $\Rightarrow$ testable with $J$-statistic.
\end{itemize}
\end{itemize}
\end{frame}

\begin{frame}{Pooled Regression Anatomy}
\begin{itemize}
\item Call $\bm{x}_{it}$ the k by 1 vector of right hand side variables. \pause
\item Grand means:
\item $\bm{\bar{\bar{x}}}\equiv \sum_{i=1}^N\sum_{t=1}^T \bm{x}_{it}$
\item $\bar{\bar{y}}\equiv \sum_{i=1}^N\sum_{t=1}^T y_{it}$\pause
\item We can write the pooled estimator as:
$$\bm{b_P}=(\bm{S}^P_{xx})^{-1}\bm{S}_{xy}^P$$
Where 
$$\bm{S}^P_{xx}= \sum_{i=1}^N\sum_{t=1}^T(\bm{x}_{it}-\bar{\bar{\bm{x}}})(\bm{x}_{it}-\bar{\bar{\bm{x}}})'$$
$$\bm{S}^P_{xy}= \sum_{i=1}^N\sum_{t=1}^T(\bm{x}_{it}-\bar{\bar{\bm{x}}})(y_{it}-\bar{\bar{y}})'$$
\end{itemize}
\end{frame}

\begin{frame}[fragile]
  \begin{lstlisting}
airlines<-airlines%>% mutate(
  mlnoutput = mean(lnoutput),
  mlnpfuel = mean(lnpfuel),
  mload = mean(load))
    
Xp<-airlines%>%transmute(
		lnoutputdm = lnoutput-mlnoutput, 
		lnpfueldm = lnpfuel-mlnpfuel,
		loaddm = load-mload)%>%
		as.matrix()

Sp_xx = t(Xp)%*%Xp
  \end{lstlisting}
\end{frame}



\begin{frame}{Between Regression}
\begin{itemize}
\item Part of the variation in $y_{it}$ and $\bm{x}_{it}$ is variation across units and part across time.\pause
\item The cross sectional part is just:
\begin{align*}
\frac{1}{T}(y_{i1}+y_{i2}+\ldots +y_{iT})&=\bm{\beta}'(\frac{1}{T}(\bm{x}_{i1}+\bm{x}_{i2}+\ldots +\bm{x}_{iT}))+\frac{1}{T}(e_{i1}+e_{i2}+\ldots+e_{iT})\\
\bar{y}_i&=\bm{\beta}'\bar{\bm{x}}_i+\bar{e}_i
\end{align*}
Define $\bm{b_B}=(\bm{S}^B_{xx})^{-1}\bm{S}_{xy}^B$.
$$\bm{S}^B_{xx}= \sum_{i=1}^N\sum_{t=1}^T(\bm{\bar{x}}_{i}-\bar{\bar{\bm{x}}})(\bm{\bar{x}}_{i}-\bar{\bar{\bm{x}}})'=\sum_{i=1}^N T(\bm{\bar{x}}_{i}-\bar{\bar{\bm{x}}})(\bm{\bar{x}}_{i}-\bar{\bar{\bm{x}}})'$$
$$\bm{S}^B_{xy}= \sum_{i=1}^N\sum_{t=1}^T(\bm{\bar{x}}_{i}-\bar{\bar{\bm{x}}})(\bar{y}_{i}-\bar{\bar{y}})=\sum_{i=1}^N T(\bm{\bar{x}}_{i}-\bar{\bar{\bm{x}}})(\bar{y}_{i}-\bar{\bar{y}})$$
\end{itemize}
\end{frame}

\begin{frame}[fragile]
  \begin{lstlisting}
airlines<-airlines%>%
  group_by(airline)%>%
  mutate(
  gmlnoutput=mean(lnoutput),
  gmlnpfuel=mean(lnpfuel),
  gmload=mean(load))
  
Xb<-airlines%>%transmute(
		blnoutputdm = gmlnoutput-mlnoutput, 
                 blnpfueldm = gmlnpfuel-mlnpfuel,
                 bloaddm = gmload-mload) %>%
                 as.matrix()
                     
  Sb_xx = t(Xb)%*%Xb
  \end{lstlisting}
\end{frame}

\begin{frame}[fragile]
  \begin{lstlisting}
airlines<-airlines%>%
  group_by(airline)%>%
  mutate(
  gmlnoutput=mean(lnoutput),
  gmlnpfuel=mean(lnpfuel),
  gmload=mean(load))
  
  Xw<-airlines%>%transmute(
  		plnoutputdm = lnoutput-gmlnoutput, 
                 plnpfueldm = lnpfuel-gmlnpfuel,
                 ploaddm = load-gmload)%>%
                 as.matrix()
                     
  Sw_xx = t(Xw)%*%Xw
  \end{lstlisting}
\end{frame}

\begin{frame}{Decomposition of Variance}
\begin{align*}
y_{it}&=\bar{y}_i+(y_{it}-\bar{y}_i)\\\pause
\bm{S}^P_{xx}&=\bm{S}^B_{xx}+\bm{S}^W_{xx}\\
\bm{S}^P_{xy}&=\bm{S}^B_{xy}+\bm{S}^W_{xy}
\end{align*}
\end{frame}


\begin{frame}{Relationship between Pooled, Between and Within}\begin{align*}
\bm{b}_P&=(\bm{S}^P_{xx})^{-1}\bm{S}^P_{xy}\\\pause
&=(\bm{S}^B_{xx}+\bm{S}^W_{xx})^{-1}(\bm{S}^B_{xy}+\bm{S}^W_{xy})\\\pause
&=(\bm{S}^B_{xx}+\bm{S}^W_{xx})^{-1}\bm{S}^W_{xy}+(\bm{S}^B_{xx}+\bm{S}^W_{xx})^{-1}\bm{S}^B_{xy}\\\pause
&=(\bm{S}^B_{xx}+\bm{S}^W_{xx})^{-1}\bm{S}^W_{xx}\bm{b}_W+(\bm{S}^B_{xx}+\bm{S}^W_{xx})^{-1}\bm{S}^B_{xx}\bm{b}_B\\\pause
&=(\bm{S}^B_{xx}+\bm{S}^W_{xx})^{-1}\bm{S}^W_{xx}\bm{b}_W+(\bm{S}^B_{xx}+\bm{S}^W_{xx})^{-1}(\bm{S}^B_{xx}+\bm{S}^W_{xx}-\bm{S}^W_{xx})\bm{b}_B\\\pause
&=(\bm{S}^B_{xx}+\bm{S}^W_{xx})^{-1}\bm{S}^W_{xx}\bm{b}_W+(\bm{I}-(\bm{S}^B_{xx}+\bm{S}^W_{xx})^{-1}\bm{S}^{W}_{xx})\bm{b}_B\\\pause
&=\bm{\lambda}\bm{b}_W+(\bm{I}-\bm{\lambda})\bm{b}_B\\
\end{align*}
\end{frame}

\begin{frame}{Relationship between Pooled, Between and Within}
\scriptsize{
\begin{table}[!htbp] \centering 
\begin{tabular}{@{\extracolsep{5pt}}lccc} 
\\[-1.8ex]\hline 
\hline \\[-1.8ex] 
\cline{2-4} 
\\[-1.8ex] & \multicolumn{3}{c}{lncost} \\ 
\\[-1.8ex] & Pooled & Between & Within\\ 
\hline \\[-1.8ex] 
 lnoutput & 0.883$^{***}$ & 0.782$^{**}$ & 0.919$^{***}$ \\ 
  & (0.013) & (0.109) & (0.030) \\ 
  & & & \\ 
 lnpfuel & 0.454$^{***}$ & $-$5.524 & 0.417$^{***}$ \\ 
  & (0.020) & (4.479) & (0.015) \\ 
  & & & \\ 
 load & $-$1.628$^{***}$ & $-$1.751 & $-$1.070$^{***}$ \\ 
  & (0.345) & (2.743) & (0.202) \\ 
  & & & \\ 
 Constant & 9.517$^{***}$ & 85.809 &  \\ 
  & (0.229) & (56.483) &  \\ 
  & & & \\ 
\hline \\[-1.8ex] 
Observations & 90 & 6 & 90 \\ 
R$^{2}$ & 0.988 & 0.994 & 0.993 \\ 
\hline 
\hline \\[-1.8ex] 
\end{tabular} 
\end{table} }
\end{frame}
%---------------------------------------------------------------
% Frame 11 — Substantive interpretation of the estimates
\begin{frame}{What the estimates tell us}
  \begin{itemize}
    \item \textbf{Output elasticity}  
      \[
        \hat\beta_1 = 0.883\;(\text{pooled}) \;\;\Longrightarrow\;\;
        \hat r = \frac{1}{\hat\beta_1} \approx 1.13.
      \] 
      The industry exhibits \emph{mild increasing returns to scale} (costs rise
      less than proportionally with output).

    \item \textbf{Fuel price elasticity}  
      \[
        \hat\beta_2 = 0.454 \;\;\Longrightarrow\;\;
        \hat\alpha_F = \hat r\,\hat\beta_2 \approx 1.13 \times 0.454 \approx 0.51.
      \]
      Roughly half of total variable cost is attributable to fuel.

    \item \textbf{Load factor effect}  
      The coefficient on \textit{load} ($-1.63$) means that a
      1-percentage-point increase in capacity utilisation lowers total cost
      by about 1.6 \%, consistent with spreading fixed operating expenses over
      more output.
  \end{itemize}
\end{frame}

\begin{frame}{Danger}
\includegraphics[width=3 in]{Danger.jpeg}\\
The left panel displays the distribution of media congruence from Snyder and Str\"{o}mberg (2010) before and after the incumbent and year fixed effects.
\end{frame}

\begin{frame}{Political Science Application: Tax Progressivity and War}
\begin{itemize}
\item \textbf{Scheve \& Stasavage (2010, 2012):} Why did top tax rates rise dramatically in the 20th century?
\item \textbf{Hypothesis:} Mass military mobilization during WWI/WWII created political pressure for progressive taxation as a ``conscription of wealth.''
\item \textbf{Data:} Country-year panel, 20 countries, 1850--1970.
\begin{itemize}
\item $Y_{it}$: Top marginal income tax rate
\item $X_{it}$: War mobilization $\times$ high-mobility indicator
\item $\alpha_i$: Country fixed effects (e.g.\ institutional differences, colonial legacies)
\end{itemize}\pause
\item \textbf{Why fixed effects?} Countries differ systematically in fiscal institutions, political systems, and economic structure. Pooled OLS would confound these with the mobilization effect.
\item Strict exogeneity plausible: war mobilization driven by geopolitics, not anticipated tax policy.
\end{itemize}
\end{frame}

\begin{frame}[fragile]{Scheve \& Stasavage: R Implementation}
\begin{lstlisting}[basicstyle=\ttfamily\scriptsize]
library(plm); library(haven); library(fixest)
dta <- read_dta("progressTax.dta")
pdta <- subset(dta, year >= 1900 & year <= 1930)

# Fixed effects with plm
fe_mod <- plm(topratep ~ wwihighmobaft,
              data = pdta,
              index = c("ccode", "year"),
              model = "within")
summary(fe_mod)

# Equivalent with fixest (+ Newey-West SEs)
fe_mod2 <- feols(topratep ~ wwihighmobaft +
    gdppcp + leftseatshp + munsuff + ratiop
    | country,
    data = pdta, panel.id = ~country + year,
    vcov = NW(1) ~ country + year)
\end{lstlisting}
\end{frame}

\begin{frame}{Summary}
\begin{itemize}
\item Difference in difference cancels out group-invariant trends and time invariant group differences (whether of interest or not).
\item Fixed effects give the same estimates as demeaning $\bm{X}$ and $\bm{y}$ by their over time means, the ``within'' estimator (so long as you correct the residual variance estimator).
\item The within estimator corresponds to the GMM moment condition $E[\tilde{\bm{X}}_{it}'\tilde{e}_{it}]=\bm{0}$, and is just-identified.
\item A ``between'' estimator only uses variation across units, averaged over time.
\item Pooled Regression is a weighted average of the between and within estimates, with weights $(\bm{S}^B_{xx}+\bm{S}^W_{xx})^{-1}\bm{S}^{W}_{xx}$.
\item \textbf{Next time:} Random effects adds between-group moments (GLS); Arellano-Bond uses lagged levels as instruments for dynamic panels --- both fit into our GMM framework from Lectures 15--16.
\end{itemize}
\end{frame}

\end{document}