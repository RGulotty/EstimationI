\documentclass[aspectratio=169, handout]{beamer}

%\usepackage[table]{xcolor}
\mode<presentation> {
\setbeamercovered{transparent}
  \usetheme{Boadilla}

\renewcommand{\familydefault}{cmss}
\usepackage{bm}
\usepackage{listings}
\useinnertheme{rectangles}
}
\usepackage{amsmath}
\usepackage{bbold}
\usepackage{tcolorbox}
\setbeamercolor{normal text}{fg=black}
\setbeamercolor{structure}{fg= blue}
\definecolor{trial}{cmyk}{1,0,0, 0}
\definecolor{trial2}{cmyk}{0.00,0,1, 0}
\definecolor{darkgreen}{rgb}{0,.4, 0.1}
\definecolor{darkpurple}{rgb}{0.4, 0, 0.6}
\usepackage{array}
\beamertemplatesolidbackgroundcolor{white}  \setbeamercolor{alerted
text}{fg=darkpurple}
\setbeamertemplate{caption}[numbered]\newcounter{mylastframe}

\font\domino=domino
\def\die#1{{\domino#1}}
\usepackage{tikz}
\usetikzlibrary{arrows}
\usepackage{colortbl}

\renewcommand{\familydefault}{cmss}

\usepackage{tikz}
\usepackage{lipsum}
\usepackage{booktabs}

\lstset{%
  language=R,
  basicstyle=\ttfamily\small,
  keywordstyle=\color{blue},
  commentstyle=\color{darkgreen},
  stringstyle=\color{darkpurple},
  showstringspaces=false,
  breaklines=true,
  frame=single,
  backgroundcolor=\color{gray!10}
}

 \newenvironment{changemargin}[3]{%
 \begin{list}{}{%
 \setlength{\topsep}{0pt}%
 \setlength{\leftmargin}{#1}%
 \setlength{\rightmargin}{#2}%
 \setlength{\topmargin}{#3}%
 \setlength{\listparindent}{\parindent}%
 \setlength{\itemindent}{\parindent}%
 \setlength{\parsep}{\parskip}%
 }%
\item[]}{\end{list}}
\usetikzlibrary{arrows}
\usetikzlibrary{arrows.meta, positioning}
\usepackage{pgfplots}
\pgfplotsset{compat=1.17}
\usecolortheme{lily}

\newtheorem{com}{Comment}
\newtheorem{lem} {Lemma}
\newtheorem{prop}{Proposition}
\newtheorem{condition}{Condition}
\newtheorem{thm}{Theorem}
\newtheorem{defn}{Definition}
\newtheorem{cor}{Corollary}
\newtheorem{obs}{Observation}
 \numberwithin{equation}{section}

\makeatletter
\def\beamerorig@set@color{%
  \pdfliteral{\current@color}%
  \aftergroup\reset@color
}
\def\beamerorig@reset@color{\pdfliteral{\current@color}}
\makeatother
\setbeamertemplate{navigation symbols}{}

\useoutertheme{miniframes}
\title[PLSC 30700]{Linear Models Lecture 13: Instrumental Variables}

\author{Robert Gulotty}
\institute[Chicago]{University of Chicago}
\vspace{0.3in}


\begin{document}

\begin{frame}
\maketitle
\end{frame}

%%%%%%%%%%%%%%%%%%%%%%%%%%%%%%%%%%%%%%%%%%%%%%%%%%%
\section{Endogeneity}
%%%%%%%%%%%%%%%%%%%%%%%%%%%%%%%%%%%%%%%%%%%%%%%%%%%

\begin{frame}{Structural vs.\ Projection Parameters}
\begin{itemize}
\item Recall from our earlier lectures: OLS estimates the \textbf{best linear predictor} (projection).
\item A \textbf{structural model} posits a causal data-generating process:
$$Y = \bm{x}'\beta + e$$
\item We can always define a projection: $\beta^*=(\mathbb{E}[\bm{x}\bm{x}'])^{-1}\mathbb{E}[\bm{x}Y]$, with $\mathbb{E}[\bm{x}e^*]=0$.
\item When $\mathbb{E}[\bm{x}e]\neq 0$:
\begin{align*}
\beta^*&=\beta+(\mathbb{E}[\bm{x}\bm{x}'])^{-1}\mathbb{E}[\bm{x}e]\neq \beta
\end{align*}
\item OLS is \alert{inconsistent} for the structural parameter:  $\hat{\beta}\xrightarrow{p}\beta^*\neq\beta$.
\item We call $\bm{x}$ \textbf{endogenous} when this occurs.
\end{itemize}
\end{frame}


%%%%%%%%%%%%%%%%%%%%%%%%%%%%%%%%%%%%%%%%%%%%%%%%%%%
\subsection{Measurement Error}
%%%%%%%%%%%%%%%%%%%%%%%%%%%%%%%%%%%%%%%%%%%%%%%%%%%

\begin{frame}{Endogeneity Source 1: Measurement Error}
\begin{itemize}
\item The true structural model is $\mathbb{E}[Y\mid \bm{z}]=\bm{z}'\beta$, but we don't observe $\bm{z}$.
\item Our measured variables: $\bm{x}=\bm{z}+\bm{u}$, where $\bm{u}$ is measurement error.
\item Assumptions on the error:
\begin{itemize}
\item $\text{plim}\;\frac{\bm{z}'\bm{u}}{n}=0$: measurement error uncorrelated with truth
\item $\text{plim}\;\frac{e'\bm{u}}{n}=0$: measurement error uncorrelated with structural disturbance
\end{itemize}
\item \textbf{Political science examples:}
\begin{itemize}
\item Survey-reported ideology (ANES self-placement on liberal--conservative scale)
\item GDP in developing countries used in aid allocation studies
\item Self-reported voter turnout (overreported by $\sim$10--15\%)
\end{itemize}
\end{itemize}
\end{frame}


\begin{frame}{Measurement Error: Rewriting in Observables}
\begin{itemize}
\item Substitute $\bm{z}=\bm{x}-\bm{u}$ into the structural equation:
\begin{align*}
Y&=\bm{z}'\beta+e\\
&=(\bm{x}-\bm{u})'\beta+e\\
&=\bm{x}'\beta+\underbrace{(e-\bm{u}'\beta)}_{\equiv\;\nu}
\end{align*}
\item But $\bm{x}$ and $\nu$ are correlated:
$$\mathbb{E}[\bm{x}\nu]=\mathbb{E}[(\bm{z}+\bm{u})(e-\bm{u}'\beta)]=-\mathbb{E}[\bm{u}\bm{u}']\beta\neq 0$$
\item The measurement error in regressors creates endogeneity, even though the true model is correctly specified.
\end{itemize}
\end{frame}


\begin{frame}{Measurement Error: Attenuation Bias}
\begin{itemize}
\item The OLS probability limit:
\begin{align*}
\text{plim}\;\hat{\beta}&=\beta+\left(\text{plim}\;\frac{\bm{X}'\bm{X}}{n}\right)^{-1}\text{plim}\;\frac{\bm{X}'\bm{u}}{n}\beta\\
&= \beta - \Sigma_X^{-1}\Sigma_{\bm{u}}\beta\\
&= \left(\bm{I}-\underbrace{\Sigma_X^{-1}}_{\text{signal}}\underbrace{\Sigma_{\bm{u}}}_{\text{noise}}\right)\beta
\end{align*}
\item In the scalar case: $\text{plim}\;\hat{\beta}=\frac{\sigma_z^2}{\sigma_z^2+\sigma_u^2}\beta$
\item OLS is biased \textbf{toward zero} --- this is \alert{attenuation bias}.
\item Even if only one variable has measurement error, it affects \textbf{all} slope coefficients.
\end{itemize}
\end{frame}


\begin{frame}{Measurement Error Example: Ideology and Voting}
\begin{itemize}
\item \textbf{Question:} Does ideological extremism reduce electoral support?
$$\text{VoteShare}_i = \beta_0 + \beta_1\;\text{Ideology}_i^* + \bm{x}_i'\gamma + e_i$$
\item $\text{Ideology}^*$ = true ideological position (unobserved).
\item We measure ideology with error:
$$\text{Ideology}_i = \text{Ideology}_i^* + u_i$$
(e.g., survey responses, roll-call scores like DW-NOMINATE).
\item Attenuation bias: OLS \textbf{underestimates} the penalty for extremism.
\item \textbf{IV solution:} Use a second measure of ideology (e.g., campaign finance scores, CFscores) as an instrument --- it shares the true signal but has independent measurement error.
\end{itemize}
\end{frame}


%%%%%%%%%%%%%%%%%%%%%%%%%%%%%%%%%%%%%%%%%%%%%%%%%%%
\subsection{Simultaneity and Endogenous Choice}
%%%%%%%%%%%%%%%%%%%%%%%%%%%%%%%%%%%%%%%%%%%%%%%%%%%

\begin{frame}{Endogeneity Source 2: Simultaneity}
\begin{itemize}
\item Two equations are jointly determined --- each variable is both cause and effect.
\item \textbf{Classic example:} Arms races (Richardson model).
\begin{align*}
\text{MilSpend}_A &= \alpha_0 + \alpha_1\;\text{MilSpend}_B + e_A\\
\text{MilSpend}_B &= \gamma_0 + \gamma_1\;\text{MilSpend}_A + e_B
\end{align*}
\item $\text{MilSpend}_B$ appears on the RHS of equation 1, but is determined in equation 2 (which depends on $\text{MilSpend}_A$).
\item OLS on either equation is inconsistent: the regressor is correlated with the error by construction.
\item \textbf{Other examples:} Trade policy and trade flows; campaign spending and vote share; policing levels and crime rates.
\end{itemize}
\end{frame}


\begin{frame}{Endogeneity Source 3: Endogenous Choice (Selection)}
\begin{itemize}
\item Agents \textbf{choose} their treatment based on expected gains (Roy 1951).
\item Potential outcomes: $Y_i(1)=\bm{x}_i'\beta_1+e_{1i}$, \; $Y_i(0)=\bm{x}_i'\beta_0+e_{0i}$.
\item Selection rule: individual chooses treatment if net benefit exceeds threshold:
$$D_i = \mathbb{1}\{\bm{z}_i'\gamma + \eta_i > 0\}, \qquad \eta_i = e_{1i}-e_{0i}$$
\item If $\eta_i$ and $e_{1i}$ are correlated:
$$\mathbb{E}[e_{1i}\mid D_i=1]=\mathbb{E}[e_{1i}\mid \bm{z}_i'\gamma+\eta_i>0]\neq 0$$
\item OLS on the treated sample is \alert{biased} --- we only observe outcomes for those who chose treatment.
\item If $\bm{z}$ includes variables excluded from $\bm{x}$, we have an \textbf{exclusion restriction}.
\end{itemize}
\end{frame}


\begin{frame}{Endogenous Choice: Political Selection}
\begin{itemize}
\item \textbf{Question:} Does holding office increase personal wealth?
$$\text{Wealth}_i = \beta_0 + \beta_1\;\text{HeldOffice}_i + \bm{x}_i'\gamma + e_i$$
\item Problem: Who runs for office? Who wins?
\begin{itemize}
\item Wealthier individuals may be more likely to run and win
\item More politically connected individuals may both win and accumulate wealth
\item Selection on gains: those who expect to profit most from office seek it out
\end{itemize}
\item Observed difference $=$ \textcolor{blue}{ATT} $+$ \textcolor{darkpurple}{Selection bias}:
$$\mathbb{E}[\text{Wealth}\mid \text{Office}=1]-\mathbb{E}[\text{Wealth}\mid \text{Office}=0]$$
\item \textbf{IV solution:} Use close election outcomes (regression discontinuity / Lee 2008) as an instrument --- winning a close race is quasi-random.
\end{itemize}
\end{frame}


\begin{frame}{Selection Bias Decomposition}
\begin{itemize}
\item The observed difference in outcomes:
\begin{multline*}
\mathbb{E}[Y\mid D\!=\!1]-\mathbb{E}[Y\mid D\!=\!0] = \underbrace{\mathbb{E}[Y(1)-Y(0)\mid D\!=\!1]}_{\text{ATT}}\\
 + \underbrace{\mathbb{E}[Y(0)\mid D\!=\!1]-\mathbb{E}[Y(0)\mid D\!=\!0]}_{\textcolor{darkpurple}{\text{Type I: Selection on levels}}}
\end{multline*}
\item If treatment effects are heterogeneous ($\tau_i = Y_i(1)-Y_i(0)$ varies):
$$\mathbb{E}[\tau_i\mid D=1]\neq \mathbb{E}[\tau_i] \quad\Rightarrow\quad \textcolor{blue}{\text{Type II: Selection on gains}}$$
\item \textbf{Type I:} Officeholders would have been wealthier anyway (baseline differences).
\item \textbf{Type II:} Those who gain most from office are most likely to seek it (differential returns).
\end{itemize}
\end{frame}


\begin{frame}{The Common Thread}
\begin{tcolorbox}[colback=blue!5, colframe=blue!50]
In all three cases --- measurement error, simultaneity, selection --- the core problem is $\mathbb{E}[\bm{x}e]\neq 0$.  The solution: find an instrument $Z$.
\end{tcolorbox}
\vspace{0.3cm}
\begin{center}
\begin{tabular}{lll}
\toprule
\textbf{Source} & \textbf{Why $\mathbb{E}[\bm{x}e]\neq 0$} & \textbf{IV strategy}\\
\midrule
Measurement error & Noise in $\bm{x}$ enters error & Second measure of $\bm{x}^*$\\
Simultaneity & $Y$ and $X$ jointly determined & Exogenous shifter of one eq.\\
Selection/omitted var. & Choice correlated with $e$ & Excluded exogenous variable\\
\bottomrule
\end{tabular}
\end{center}
\end{frame}


%%%%%%%%%%%%%%%%%%%%%%%%%%%%%%%%%%%%%%%%%%%%%%%%%%%
\section{IV Setup}
%%%%%%%%%%%%%%%%%%%%%%%%%%%%%%%%%%%%%%%%%%%%%%%%%%%

\begin{frame}{Notation: Structural Equation (Hansen Ch.\ 12)}
\begin{itemize}
\item $Y$ is a linear function of exogenous variables $\bm{x}_1$ and endogenous variables $\bm{y}_2$:
$$Y_1=\bm{x}_1'\beta_1+\bm{y}_2'\beta_2+e, \qquad \mathbb{E}[\bm{y}_2 e]\neq 0$$
\item \textbf{Instruments:} $\bm{z}=(\bm{z}_1', \bm{z}_2')'$, dimension $l\times 1$:
\begin{itemize}
\item $\bm{z}_1 = \bm{x}_1$: included instruments (the exogenous regressors), dimension $k_1$
\item $\bm{z}_2$: excluded instruments, dimension $l_2 \geq k_2$
\end{itemize}
\item Writing $\bm{x}=(\bm{x}_1', \bm{y}_2')'$ of dimension $k=k_1+k_2$:
$$Y_1 = \bm{x}'\beta + e$$
\end{itemize}
\end{frame}


\begin{frame}{Instrumental Variable Conditions}
\begin{tcolorbox}[colback=blue!5, colframe=blue!50]
\textbf{Three conditions} for valid instruments $\bm{z}$:
\begin{enumerate}
\item \textbf{Exogeneity:}  $\mathbb{E}[\bm{z}e]=0$ \quad (instrument uncorrelated with structural error)
\item \textbf{Relevance:}  $\text{rank}\;\mathbb{E}[\bm{z}\bm{x}']=k$ \quad (instruments predict endogenous regressors)
\item \textbf{Order condition:}  $l\geq k$ \quad (at least as many instruments as regressors)
\end{enumerate}
\end{tcolorbox}
\vspace{0.2cm}
\begin{itemize}
\item When $l=k$: \textbf{just identified} (exactly as many instruments as regressors)
\item When $l>k$: \textbf{overidentified} ($q=l-k$ overidentifying restrictions)
\item When $l<k$: \textbf{underidentified} (cannot estimate $\beta$)
\end{itemize}
\end{frame}


\begin{frame}{Example: Labeling the IV Setup}
\textbf{Do democratic institutions cause economic growth?} (Acemoglu, Johnson \& Robinson 2001)
\vspace{0.2cm}
\begin{itemize}
\item \textbf{Structural equation} ($Y_1 = \bm{x}_1'\beta_1 + \bm{y}_2'\beta_2 + e$):
$$\text{GDP/capita}_i = \beta_1\;\text{Latitude}_i + \beta_2\;\underbrace{\text{Institutions}_i}_{\bm{y}_2\text{: endogenous}} + e_i$$
\item \textbf{Endogeneity:} Richer countries may adopt better institutions (reverse causality); omitted factors (geography, culture) affect both.
\item \textbf{Excluded instrument} ($\bm{z}_2$): Colonial settler mortality.
\begin{itemize}
\item \textbf{Relevance:} High settler mortality $\Rightarrow$ extractive colonies $\Rightarrow$ weak institutions today.
\item \textbf{Exogeneity:} Historical disease environment affects current GDP only through institutions (exclusion restriction --- debatable!).
\end{itemize}
\end{itemize}
\end{frame}


%%%%%%%%%%%%%%%%%%%%%%%%%%%%%%%%%%%%%%%%%%%%%%%%%%%
\subsection{Reduced Form}
%%%%%%%%%%%%%%%%%%%%%%%%%%%%%%%%%%%%%%%%%%%%%%%%%%%

\begin{frame}{Reduced Form: Definitions}
\begin{itemize}
\item The \textbf{reduced form} expresses all endogenous variables as functions of instruments only.
\item For the endogenous regressors $\bm{y}_2$ ($k_2\times 1$):
$$\bm{y}_2 = \Gamma'\bm{z}+\bm{u}_2 = \Gamma_{12}'\bm{z}_1+\Gamma_{22}'\bm{z}_2+\bm{u}_2, \qquad \mathbb{E}[\bm{z}\bm{u}_2']=0$$
where $\Gamma$ is $l\times k_2$, defined by $\Gamma = \mathbb{E}[\bm{z}\bm{z}']^{-1}\mathbb{E}[\bm{z}\bm{y}_2']$.
\item The full projection of all regressors $\bm{x}=(\bm{x}_1', \bm{y}_2')'$ on $\bm{z}$:
$$\bar{\Gamma} = \begin{bmatrix} \bm{I}_{k_1} & \Gamma_{12}\\ \bm{0} & \Gamma_{22}\end{bmatrix} = \left[\begin{matrix} \bm{I}_{k_1}\\ \bm{0}\end{matrix}\quad \Gamma\right] = \mathbb{E}[\bm{z}\bm{z}']^{-1}\mathbb{E}[\bm{z}\bm{x}']$$
\item Key: OLS consistently estimates $\Gamma$ and $\bar{\Gamma}$ because $\bm{z}$ is exogenous.
\end{itemize}
\end{frame}


\begin{frame}{Reduced Form for $Y$}
\begin{itemize}
\item Plugging the reduced form for $\bm{y}_2$ into the structural equation:
\begin{align*}
Y_1 &= \bm{z}_1'\beta_1 + (\Gamma_{12}'\bm{z}_1 + \Gamma_{22}'\bm{z}_2+\bm{u}_2)'\beta_2 + e\\
&= \bm{z}_1'\underbrace{(\beta_1+\Gamma_{12}\beta_2)}_{\lambda_1} + \bm{z}_2'\underbrace{\Gamma_{22}\beta_2}_{\lambda_2} + \underbrace{(\bm{u}_2'\beta_2+e)}_{u_1}\\
&= \bm{z}'\lambda + u_1
\end{align*}
\item The \textbf{structural} parameters are $\beta_1, \beta_2$.
\item The \textbf{reduced form} parameters are $\lambda, \Gamma$.
\item Relationship: $\lambda = \bar{\Gamma}\beta$.
\end{itemize}
\end{frame}


\begin{frame}{Reduced Form: The AJR Example}
\begin{itemize}
\item \textbf{Structural equation} (what we want):
$$\text{GDP}_i = \beta_1\;\text{Lat}_i + \beta_2\;\text{Institutions}_i + e_i$$
\item \textbf{First stage} (reduced form for $\bm{y}_2$):
$$\text{Institutions}_i = \underbrace{\Gamma_{12}}_{\text{Lat coeff}}\;\text{Lat}_i + \underbrace{\Gamma_{22}}_{\text{Mortality coeff}}\;\text{SettlerMort}_i + u_{2i}$$
\item \textbf{Reduced form for $Y$}:
$$\text{GDP}_i = \underbrace{\lambda_1}_{=\beta_1+\Gamma_{12}\beta_2}\;\text{Lat}_i + \underbrace{\lambda_2}_{=\Gamma_{22}\beta_2}\;\text{SettlerMort}_i + u_{1i}$$
\item $\lambda_2$ is the ``reduced form effect'' of settler mortality on GDP.
\item The structural parameter: $\beta_2 = \lambda_2 / \Gamma_{22}$ \quad (ratio of reduced form to first stage).
\end{itemize}
\end{frame}


\begin{frame}{Identification}
\begin{itemize}
\item $\beta$ is \textbf{identified} if it is the unique solution to the moment conditions:
$$\mathbb{E}[\bm{z}(Y_1-\bm{x}'\beta)]=0$$
\item This is a system of $l$ equations in $k$ unknowns.
\item \textbf{Just identified} ($l=k$): unique solution $\beta = (\mathbb{E}[\bm{z}\bm{x}'])^{-1}\mathbb{E}[\bm{z}Y_1]$
\item Equivalently: if $\bar{\Gamma}$ has rank $k$, then $\beta=(\bar{\Gamma}'\bar{\Gamma})^{-1}\bar{\Gamma}'\lambda$.
\item \textbf{Overidentified} ($l>k$): system is overdetermined.
\begin{itemize}
\item No exact solution in general --- we need a method to combine the moment conditions.
\item \alert{Foreshadowing:} this is exactly the problem GMM solves (Lecture 15).
\end{itemize}
\item The \textbf{relevance condition} (rank $\mathbb{E}[\bm{z}\bm{x}']=k$) is what makes the solution exist.
\end{itemize}
\end{frame}


%%%%%%%%%%%%%%%%%%%%%%%%%%%%%%%%%%%%%%%%%%%%%%%%%%%
\section{The IV Estimator}
%%%%%%%%%%%%%%%%%%%%%%%%%%%%%%%%%%%%%%%%%%%%%%%%%%%

\begin{frame}{IV Estimator: Just-Identified Case}
\begin{itemize}
\item When $l=k$, the \textbf{IV estimator} is the sample analogue of the moment condition:
$$\hat{\beta}_{IV} = (Z'X)^{-1}Z'Y$$
\item Decompose:
\begin{align*}
\hat{\beta}_{IV} &= (Z'X)^{-1}Z'(X\beta+e)\\
&= \beta + (Z'X)^{-1}Z'e
\end{align*}
\item Consistency: $\text{plim}\;\hat{\beta}_{IV} = \beta + (E[ZX'])^{-1}E[Ze] = \beta$ \quad under exogeneity.
\end{itemize}
\end{frame}


\begin{frame}{Indirect Least Squares}
\begin{itemize}
\item \textbf{ILS}: Estimate reduced forms by OLS, then recover $\beta$.
\item Reduced form estimates:
$$\hat{\Gamma} = (Z'Z)^{-1}Z'X, \qquad \hat{\lambda} = (Z'Z)^{-1}Z'Y$$
\item When $l=k$: $\hat{\beta}_{ILS} = \hat{\Gamma}^{-1}\hat{\lambda}$
\item Show equivalence:
\begin{align*}
\hat{\beta}_{ILS} &= [(Z'Z)^{-1}Z'X]^{-1}(Z'Z)^{-1}Z'Y\\
&= (Z'X)^{-1}(Z'Z)(Z'Z)^{-1}Z'Y\\
&= (Z'X)^{-1}Z'Y = \hat{\beta}_{IV}
\end{align*}
\item ILS = IV in the just-identified case.
\end{itemize}
\end{frame}


\begin{frame}{The Wald Estimator}
\begin{itemize}
\item Special case: single endogenous $X$, single binary instrument $Z\in\{0,1\}$.
\item The IV estimator simplifies to the \textbf{Wald estimator}:
$$\hat{\beta}_{Wald} = \frac{\bar{Y}_{Z=1}-\bar{Y}_{Z=0}}{\bar{X}_{Z=1}-\bar{X}_{Z=0}} = \frac{\text{Reduced form effect on } Y}{\text{First stage effect on } X}$$
\item Intuition: scale the \textbf{intent-to-treat} (ITT) effect by the first-stage compliance rate.
\item This is a ratio of two consistent estimators --- a \textbf{ratio estimator}.
\end{itemize}
\begin{tcolorbox}[colback=blue!5, colframe=blue!50]
The Wald/IV estimator is the ratio of the reduced form to the first stage.  This ``ratio'' structure is central to understanding IV.
\end{tcolorbox}
\end{frame}


\begin{frame}{Wald Estimator: Draft Lottery and Civic Participation}
\begin{itemize}
\item \textbf{Question:} Does military service increase political participation?
\item $Y$ = voter turnout, $X$ = served in military, $Z$ = draft lottery number (low = drafted).
\item \textbf{Reduced form:} $\bar{Y}_{Z=\text{low}} - \bar{Y}_{Z=\text{high}}$ = effect of lottery on turnout (ITT).
\item \textbf{First stage:} $\bar{X}_{Z=\text{low}} - \bar{X}_{Z=\text{high}}$ = effect of lottery on service rate.
$$\hat{\beta}_{Wald} = \frac{\text{ITT on turnout}}{\text{First stage compliance}} = \frac{\text{Reduced form}}{\text{First stage}}$$
\item Not everyone drafted actually serves (deferments, exemptions).
\item The Wald estimator rescales the ITT by the share who comply with the draft.
\end{itemize}
\end{frame}


%%%%%%%%%%%%%%%%%%%%%%%%%%%%%%%%%%%%%%%%%%%%%%%%%%%
\section{2SLS}
%%%%%%%%%%%%%%%%%%%%%%%%%%%%%%%%%%%%%%%%%%%%%%%%%%%

\begin{frame}{Two-Stage Least Squares}
\begin{itemize}
\item When $l>k$ (overidentified), we need 2SLS.
\item \textbf{Stage 1:}  Regress $X$ on $Z$:
$$\hat{X} = P_Z X, \qquad P_Z = Z(Z'Z)^{-1}Z'$$
\item \textbf{Stage 2:}  Regress $Y$ on $\hat{X}$:
$$\hat{\beta}_{2SLS} = (\hat{X}'\hat{X})^{-1}\hat{X}'Y = (X'P_ZX)^{-1}X'P_ZY$$
\item The projection $P_Z$ extracts the \textbf{exogenous variation} in $X$ --- the part predicted by instruments.
\end{itemize}
\end{frame}


\begin{frame}{2SLS = IV When Just Identified}
\begin{itemize}
\item When $l=k$, $P_Z = Z(Z'Z)^{-1}Z'$ and:
\begin{align*}
\hat{\beta}_{2SLS}&=(\hat{X}'\hat{X})^{-1}\hat{X}'Y\\
&=(X'P_ZX)^{-1}X'P_ZY\\
&=(X'Z(Z'Z)^{-1}Z'X)^{-1}X'Z(Z'Z)^{-1}Z'Y\\
&=(Z'X)^{-1}(Z'Z)(X'Z)^{-1}X'Z(Z'Z)^{-1}Z'Y \tag{$(ABC)^{-1}=C^{-1}B^{-1}A^{-1}$}\\
&=(Z'X)^{-1}Z'Y = \hat{\beta}_{IV}
\end{align*}
\item 2SLS generalizes IV to the overidentified case.
\end{itemize}
\end{frame}


\begin{frame}[fragile]{R Implementation}
\begin{lstlisting}
library(estimatr)

# 2SLS estimation
# Y ~ endogenous + exogenous | excluded_instruments + exogenous
fit_iv <- iv_robust(Y ~ D + X | Z + X, data = dat)
summary(fit_iv)
\end{lstlisting}
\begin{itemize}
\item Variables before ``\textbar'' are the structural equation regressors.
\item Variables after ``\textbar'' are the instruments (excluded + included).
\item \texttt{iv\_robust} uses heteroskedasticity-robust standard errors by default.
\end{itemize}
\end{frame}


%%%%%%%%%%%%%%%%%%%%%%%%%%%%%%%%%%%%%%%%%%%%%%%%%%%
\section{Asymptotic Theory}
%%%%%%%%%%%%%%%%%%%%%%%%%%%%%%%%%%%%%%%%%%%%%%%%%%%

\begin{frame}{Consistency of 2SLS (Hansen Thm.\ 12.1)}
\begin{thm}[Consistency]
Under the conditions: (1) $E[Ze]=0$, (2) $\text{rank}\;E[ZX']=k$, (3) $E[ZZ']>0$, the 2SLS estimator is consistent: $\hat{\beta}_{2SLS}\xrightarrow{p}\beta$.
\end{thm}
\vspace{0.2cm}
\textbf{Proof sketch:}
$$\hat{\beta}_{2SLS}-\beta = \left(\frac{X'P_ZX}{n}\right)^{-1}\frac{X'P_Ze}{n}$$
Define $Q_{XZ}=E[XZ']$, $Q_{ZZ}=E[ZZ']$, $Q_{ZX}=E[ZX']$.  Then:
$$\xrightarrow{p}\; (Q_{XZ}Q_{ZZ}^{-1}Q_{ZX})^{-1}Q_{XZ}Q_{ZZ}^{-1}\underbrace{E[Z'e]}_{=0} = 0$$
\end{frame}


\begin{frame}{Asymptotic Distribution (Hansen Thm.\ 12.2)}
\begin{thm}[Asymptotic Normality]
Under regularity conditions:
$$\sqrt{n}(\hat{\beta}_{2SLS}-\beta)\xrightarrow{d} N(0, V_\beta)$$
where:
$$V_\beta = (Q_{XZ}Q_{ZZ}^{-1}Q_{ZX})^{-1}Q_{XZ}Q_{ZZ}^{-1}\Omega Q_{ZZ}^{-1}Q_{ZX}(Q_{XZ}Q_{ZZ}^{-1}Q_{ZX})^{-1}$$
with $Q_{XZ}=E[XZ']$, $Q_{ZZ}=E[ZZ']$, $Q_{ZX}=E[ZX']$, $\Omega=E[Z'Zee'Z]$.
\end{thm}
\begin{itemize}
\item Under homoskedasticity ($E[e^2\mid Z]=\sigma^2$), this simplifies to:
$$V_\beta = \sigma^2(Q_{XZ}Q_{ZZ}^{-1}Q_{ZX})^{-1}$$
\item Under heteroskedasticity, use robust covariance estimation (as in Lecture 8).
\end{itemize}
\end{frame}


\begin{frame}{Variance Estimation}
\begin{itemize}
\item \textbf{Homoskedastic} variance estimator:
$$\hat{V}_\beta = \hat{\sigma}^2(X'P_ZX)^{-1}, \qquad \hat{\sigma}^2 = \frac{\hat{e}'\hat{e}}{n-k}$$
where $\hat{e}=Y-X\hat{\beta}_{2SLS}$ (use \textbf{original} $X$, not $\hat{X}$).
\item \textbf{Heteroskedastic-robust} (analogous to HC):
$$\hat{V}_\beta = (X'P_ZX)^{-1}\left(\sum_{i=1}^n \hat{e}_i^2 \hat{X}_i\hat{X}_i'\right)(X'P_ZX)^{-1}$$
\item \alert{Important:} Standard errors from the ``naive'' second-stage regression (regressing $Y$ on $\hat{X}$ and reading off SEs) are \textbf{incorrect} --- they use $\hat{X}$ instead of $X$ in the residuals.
\end{itemize}
\end{frame}


%%%%%%%%%%%%%%%%%%%%%%%%%%%%%%%%%%%%%%%%%%%%%%%%%%%
\section{Moment Conditions and GMM Preview}
%%%%%%%%%%%%%%%%%%%%%%%%%%%%%%%%%%%%%%%%%%%%%%%%%%%

\begin{frame}{IV as a Method of Moments}
\begin{itemize}
\item The IV estimator solves the \textbf{sample moment condition}:
$$\frac{1}{n}\sum_{i=1}^n Z_i(Y_i - X_i'\beta) = 0$$
\item This is a system of $l$ equations in $k$ unknowns.
\item When $l=k$: exactly identified, unique solution.
\item When $l>k$: overidentified, no exact solution.
\item \textbf{2SLS resolves this} by using a specific weighting matrix:
$$\hat{\beta}_{2SLS} = \arg\min_\beta \left[\frac{1}{n}\sum Z_i(Y_i-X_i'\beta)\right]'(Z'Z/n)^{-1}\left[\frac{1}{n}\sum Z_i(Y_i-X_i'\beta)\right]$$
\end{itemize}
\end{frame}


\begin{frame}{Foreshadowing GMM}
\begin{itemize}
\item 2SLS uses the weighting matrix $\hat{W}=(Z'Z/n)^{-1}$.
\item Is this the \textbf{optimal} weighting?
\begin{itemize}
\item Under homoskedasticity: \textbf{yes}.
\item Under heteroskedasticity: \textbf{no} --- there exists a more efficient choice.
\end{itemize}
\item The \textbf{Generalized Method of Moments} (GMM) chooses $\hat{W}$ optimally:
$$\hat{W}_{opt} = \hat{\Omega}^{-1}, \qquad \hat{\Omega} = \frac{1}{n}\sum_{i=1}^n \hat{e}_i^2 Z_iZ_i'$$
\item GMM also extends beyond linear IV to \textbf{any} set of moment conditions.
\item \alert{Next week:} we develop the full GMM framework.
\end{itemize}
\end{frame}


\begin{frame}{Summary}
\begin{enumerate}
\item Endogeneity ($E[Xe]\neq 0$) makes OLS inconsistent for structural parameters.
\item Instruments $Z$ satisfying exogeneity and relevance allow consistent estimation.
\item The IV estimator $\hat{\beta}_{IV}=(Z'X)^{-1}Z'Y$ works when just identified.
\item 2SLS $\hat{\beta}_{2SLS}=(X'P_ZX)^{-1}X'P_ZY$ generalizes to overidentification.
\item Both are consistent and asymptotically normal under standard conditions.
\item IV is a \textbf{method of moments} --- 2SLS is a specific weighting of moment conditions.
\end{enumerate}
\vspace{0.2cm}
\begin{tcolorbox}[colback=blue!5, colframe=blue!50]
\textbf{Next lecture:} Finite sample properties, testing, LATE, and the MTE framework.
\end{tcolorbox}
\end{frame}

\end{document}
