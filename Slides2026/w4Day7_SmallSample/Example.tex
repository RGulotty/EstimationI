\documentclass[tikz]{standalone}
\usepackage{amsmath, amssymb}
\usepackage{tikz}
\usetikzlibrary{calc}

% Define a custom normal pdf function usable in TikZ


\begin{document}
\begin{tikzpicture}[scale=0.9, font=\small]

% Axis labels
\draw[->] (0.5, -2.5) -- (5.5, -2.5) node[right] {X};
\draw[->] (0.5, -2.5) -- (0.5, 3.5) node[above] {Y*};

% Vertical dotted lines at X1..X9
\foreach \i in {1,...,5} {
  \draw[dotted] (\i, -2.5) -- (\i, 3.5);
  \node[below] at (\i, -2.5) {$X_{\i}$};
}

% Threshold: dashed horizontal line at y=0
\draw[dashed] (0.5, 0) -- (5.5, 0) node[pos=1,right] {Threshold};

% We'll assume a linear \"mean\" function: mu_i = intercept + slope*i
% Example: intercept = -2, slope = 0.6
\def\intercept{-2}
\def\slope{0.6}
\def\sigma{0.7}   % std dev for each normal
\def\hscale{0.5} % horizontal scaling of the pdf curves

% Draw a black line connecting the means from i=1 to i=9
\draw[thick]
  (1, {\intercept + \slope*1})
  -- (5, {\intercept + \slope*5});

% Loop over i=1..9 to draw each vertical normal distribution
\foreach \i in {1,...,5} {
  % Compute mean = intercept + slope * i
  \pgfmathsetmacro{\mu}{\intercept + \slope*\i}

  % 1) Draw the pdf curve
  \draw[domain=-3:3, samples=50, smooth, variable=\z, blue]
    plot
    (
      {\i + \hscale * normalpdf(\z,0,1)}, % x-coordinate
      {\z * \sigma + \mu}                 % y-coordinate
    );

  % 2) Shade the area above the threshold (y=0)
  \begin{scope}
    % Clip region: only keep anything above y=0 near x=\i
    \clip (\i-0.5, 0) rectangle (\i+1, 3.5);
    \path[fill=blue!20]
      plot[domain=-3:3, samples=50, smooth, variable=\z]
      (
        {\i + \hscale * normalpdf(\z,0,1)},
        {\z * \sigma + \mu}
      )
      -- ++(0,-5) -- cycle; % connect down below to close the shape
  \end{scope}
}

\end{tikzpicture}
\end{document}
